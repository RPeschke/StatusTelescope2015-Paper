\pdfoutput=1 % only if pdf/png/jpg images are used
\documentclass{JINST}

\usepackage{graphics}           % Unterstuetzung fuer Grafiken (extended)
\usepackage{graphicx}
\usepackage{subfigure}
\usepackage{amsmath}
\usepackage{amssymb}
\usepackage{graphics}           % Unterstuetzung fuer Grafiken (extended)
\usepackage[load-configurations=abbreviations,tight-spacing=true,separate-uncertainty,bracket-numbers = false]{siunitx}
\usepackage{nicefrac}
\usepackage[english]{babel}
\usepackage{color}
\usepackage{lineno}
\usepackage{epstopdf}
\usepackage{stfloats}

\newcommand{\e}{\ensuremath{\mathnormal{e}}}
\newcommand{\h}{\ensuremath{\mathnormal{h}}}
\newcommand{\Datura}{\ensuremath{\mathnormal{DATURA}}}

\DeclareSIUnit\e{\ensuremath{\mathnormal{e}}}
\DeclareSIUnit\sample{\ensuremath{\mathrm{S}}}
\DeclareSIUnit\ppb{\ensuremath{\mathrm{ppb}}}
\DeclareSIUnit\GeV{\ensuremath{\mathrm{GeV}}}


\title{Status of the DATURA Telescope 2015}
\author{A. Author${}^{\textrm{a},}$\thanks{corresponding author, \textit{Email:} my.mail@desy.cde}, B. Author${}^{\textrm{b}}$}

\abstract{
The status of the $\Datura$ Telescope at DESY is summarised and the performance is shown with two example studies. 
The pointing resolution using a $\SI{6}{\GeV}$ $\e$-beam at the centre of the telescope is $\SI{5}{\micro\m}$.}


\begin{document}
 \setpagewiselinenumbers
\modulolinenumbers[5]
\linenumbers


\date{\today}
\maketitle


%\small
\hspace{0.85cm}${}^{\textrm{a}}$ DESY, Hamburg, Germany, ${}^{\textrm{b}}$ Institute for Telescopocy, bla, blubb


\normalsize

\section{Introduction}
 
This is an introduction.
This is a citation.\,\cite{PDG}

\section{Desciprion of the Telescope Hardware}
%\input{HW}

\section{Desciprion of the Telescope Software}
%\input{SW}

\section{Conclusion}

\section*{Acknowledgement}

\small
\bibliographystyle{elsarticle-num}
\bibliography{./bibtex/refs}

\end{document}


