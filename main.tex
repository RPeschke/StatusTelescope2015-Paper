\pdfoutput=1 % only if pdf/png/jpg images are used
\documentclass{JINST}

\usepackage{graphics}           % Unterstuetzung fuer Grafiken (extended)
\usepackage{graphicx}
\usepackage{subfigure}
\usepackage{amsmath}
\usepackage[amssymb]{SIunits}
%\usepackage[load-configurations=abbreviations,tight-spacing=true,separate-uncertainty,bracket-numbers = false]{siunitx}
\usepackage{nicefrac}
\usepackage[english]{babel}
\usepackage{lineno}
\usepackage{epstopdf}
\usepackage{stfloats}
\usepackage{upgreek}

\newcommand{\e}{\ensuremath{\mathnormal{e}}}
\newcommand{\h}{\ensuremath{\mathnormal{h}}}
\newcommand{\Datura}{\ensuremath{\mathnormal{DATURA}}}
\newcommand{\eV}{\ensuremath{\mathnormal{eV}}}

 \setpagewiselinenumbers
\modulolinenumbers[5]
\linenumbers

\title{Status of the DATURA Telescope 2015}
\author{A. Author${}^{\textrm{a},}$, B. Author${}^{\textrm{b}}$\\
${}^{\textrm{a}}$ Deutsches Elektronen-Synchrotron DESY, Hamburg, Germany,\\
${}^{\textrm{b}}$ Institute for Telescopocy, bla, blubb
}

\abstract{
The status of the $\Datura$ Telescope at DESY is summarised and the performance is shown with two example studies. 
The pointing resolution using a 6\,GeV $\e$-beam at the centre of the telescope is $\unit{5}{\upmu\meter}$.}


\begin{document}
 \setpagewiselinenumbers
\modulolinenumbers[5]
\linenumbers

\normalsize

\section{Introduction}
 
This is an introduction.
This is a citation.\,\cite{PDG}

\section{Introduction}

\section{Beamlines}

\section{Beam Telescope}
telescope in general, layout, mechanics, sensors

\section{Data Acquisition Framework}
euDAQ

\section{Trigger Logic}
TLU, 2TLU-Setup

\section{Offline Analysis and Reconstruction}
Eutelescope, Data analysis flow

\section{Track Resolution}

\section{Conclusion}

\section*{Acknowledgement}

\small
\bibliographystyle{plain}
\bibliography{bibtex/refs}

\end{document}


